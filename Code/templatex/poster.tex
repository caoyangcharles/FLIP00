%%
%% This is file `tikzposter-template.tex',
%% generated with the docstrip utility.
%%
%% The original source files were:
%%
%% tikzposter.dtx  (with options: `tikzposter-template.tex')
%%
%% This is a generated file.
%%
%% Copyright (C) 2014 by Pascal Richter, Elena Botoeva, Richard Barnard, and Dirk Surmann
%%
%% This file may be distributed and/or modified under the
%% conditions of the LaTeX Project Public License, either
%% version 2.0 of this license or (at your option) any later
%% version. The latest version of this license is in:
%%
%% http://www.latex-project.org/lppl.txt
%%
%% and version 2.0 or later is part of all distributions of
%% LaTeX version 2013/12/01 or later.
%%


\documentclass{tikzposter} %Options for format can be included here

\usepackage{todonotes}

\usepackage[tikz]{bclogo}
\usepackage{lipsum}
\usepackage{amsmath}

\usepackage{booktabs}
\usepackage{longtable}
\usepackage[absolute]{textpos}
\usepackage[it]{subfigure}
\usepackage{graphicx}
\usepackage{cmbright}
%\usepackage[default]{cantarell}
%\usepackage{avant}
%\usepackage[math]{iwona}
\usepackage[math]{kurier}
\usepackage[T1]{fontenc}


%% add your packages here
\usepackage{hyperref}
% for random text
\usepackage{lipsum}
\usepackage[english]{babel}
\usepackage[pangram]{blindtext}

\colorlet{backgroundcolor}{blue!10}

 % Title, Author, Institute
\title{PUBG Game Data Analysis and prediction
}
\author{Yang Cao}
\institute{Deakin University, Australia
}
%\titlegraphic{logos/tulip-logo.eps}

%Choose Layout
\usetheme{Wave}

%\definebackgroundstyle{samplebackgroundstyle}{
%\draw[inner sep=0pt, line width=0pt, color=red, fill=backgroundcolor!30!black]
%(bottomleft) rectangle (topright);
%}
%
%\colorlet{backgroundcolor}{blue!10}

\begin{document}


\colorlet{blocktitlebgcolor}{blue!23}

 % Title block with title, author, logo, etc.
\maketitle

\begin{columns}
 % FIRST column
\column{0.5}% Width set relative to text width

%%%%%%%%%% -------------------------------------------------------------------- %%%%%%%%%%
 %\block{Main Objectives}{
%  	      	\begin{enumerate}
%  	      	\item Formalise research problem by extending \emph{outlying aspects mining}
%  	      	\item Proposed \emph{GOAM} algorithm is to solve research problem
%  	      	\item Utilise pruning strategies to reduce time complexity
%  	      	\end{enumerate}
%%  	      \end{minipage}
%}
%%%%%%%%%% -------------------------------------------------------------------- %%%%%%%%%%


%%%%%%%%%% -------------------------------------------------------------------- %%%%%%%%%%
\block{Introduction}{
    In a PUBG game, up to 100 players start in each match (matchId). Players can be on teams (groupId) which get ranked at the end of the game
    (winPlacePerc) based on how many other teams are still alive when they
    are eliminated. In game, players can pick up different munitions, revive
    downed-but-not-out (knocked) teammates, drive vehicles, swim, run, shoot,
    and experience all of the consequences – such as falling too far or running
    themselves over and eliminating themselves. Different game behaviors will
    lead to different final rankings, so the main purpose is to build a model to
    predicts players’ finishing placement based on their final stats, on a scale
    from 1 (first place) to 0 (last place).
  	
  	\begin{description}
  	\item[PUBG Game Data Analysis] aims to make A game team data know which game actions that
    make game teams get higher rank than others
  	
  	\item[PUBG win place prediction] aims to help Players can also estimate their final ranking based on the current situation and make strategic decisions in advance (such as running away or
    fighting)
    Hence,
    the outlying aspects mining is also referred to
    \emph{outlier interpretation}
    or \emph{object explanation}.
  	\end{description}
}
%%%%%%%%%% -------------------------------------------------------------------- %%%%%%%%%%


%%%%%%%%%% -------------------------------------------------------------------- %%%%%%%%%%
\block{Description and evaluation}{
\begin{itemize}
    \item
    %\emph{Group Outlying Aspects Mining}
    Use Mean Square Error to evaluate model (the average squared difference between the estimated
    values and the actual value)
    \item 
    Train data MSE
    \item 
    Test Data MSE

    % \item
    % \emph{Group Outlying Aspects Mining},
    % \emph{Outlying Aspects Mining} and
    % \emph{Outlier Detection} are different with each other.
\end{itemize}

% \begin{center}
%     \begin{minipage}{0.3\linewidth}
%     \centering
%     \begin{tikzfigure}
%     \missingfigure[figcolor=white]{Testing figcolor}
%     {\small{Group Outlying Aspects Mining}}
%     \end{tikzfigure}%
%     \end{minipage}
%     \hfill
%     \begin{minipage}{0.3\linewidth}
%     \centering
%     \begin{tikzfigure}
%     \missingfigure[figcolor=white]{Testing figcolor}
%     {\small{Outlying Aspects Mining}}
%     \end{tikzfigure}%
%     \end{minipage}
%     \hfill
%     \begin{minipage}{0.3\linewidth}
%     \centering
%     \begin{tikzfigure}
%     \missingfigure[figcolor=white]{Testing figcolor}
%     {\small{Outlier Detection}}
%     \end{tikzfigure}%
%     \end{minipage}
% \end{center}
}
%%%%%%%%%% -------------------------------------------------------------------- %%%%%%%%%%


%%%%%%%%%% -------------------------------------------------------------------- %%%%%%%%%%

%\note{Note with default behavior}

%\note[targetoffsetx=12cm, targetoffsety=-1cm, angle=20, rotate=25]
%{Note \\ offset and rotated}

 % First column - second block


%%%%%%%%%% -------------------------------------------------------------------- %%%%%%%%%%
\block{Data Visualization}{
  	We propose the use data visualization technique to show the game types proportion and the relationship between 
    walking distance and win place.
%    1) Group Feature Extraction,
%    2) Outlying Degree Scoring, and
%    3) Outlying Aspects Identification.
  	
% \begin{tikzfigure}%[Overall architecture of \emph{GOAM} algorithm]
% %  \includegraphics[width=0.8\linewidth]{figures//framework.pdf}
%     \missingfigure[figcolor=white]{Testing figcolor}
% \end{tikzfigure}
		
% \begin{description}
%   	\item[Group Feature Extraction]
%   	Let $f_1$, $f_2$, $f_3$ represent three features of $G_q$.
%     We count the frequency of each value for one feature.
%     Then use the histogram to represent each feature.
%     Similarly,
%     we can extract other features for each group.

% %    \item
% %    The histogram of $G_q$ on three features are as follows.
% \end{description}

% \begin{center}
%     \begin{minipage}{0.3\linewidth}
%     \centering
%     \begin{tikzfigure}
%     \missingfigure[figcolor=white]{Testing figcolor}
%     {\small{Histogram of $G_q$ on $f_1$}}
%     \end{tikzfigure}%
%     \end{minipage}
%     \hfill
%     \begin{minipage}{0.3\linewidth}
%     \centering
%     \begin{tikzfigure}
%     \missingfigure[figcolor=white]{Testing figcolor}
%     {\small{Histogram of $G_q$ on $f_2$}}
%     \end{tikzfigure}%
%     \end{minipage}
%     \hfill
%     \begin{minipage}{0.3\linewidth}
%     \centering
%     \begin{tikzfigure}
%     \missingfigure[figcolor=white]{Testing figcolor}
%     {\small{Histogram of $G_q$ on $f_3$}}
%     \end{tikzfigure}%
%     \end{minipage}
% \end{center}
% \begin{description}
% \item[Outlying Degree Scoring]
%     In this step,
%     we first calculate the \emph{earth mover distance} (EMD) of one feature among different groups.
%     The earth mover distance reflects the minimum mean distance
%     between groups on one feature.
%     So,
%     we utilize the EMD to measure the difference between groups of each feature.
% \end{description}
}
%%%%%%%%%% -------------------------------------------------------------------- %%%%%%%%%%


% SECOND column
\column{0.5}
 %Second column with first block's top edge aligned with with previous column's top.

%%%%%%%%%% -------------------------------------------------------------------- %%%%%%%%%%
\block{Data Preprocess}{
\begin{description}
    \item
    We remove all the rows which has missing values or NaN values.
\end{description}

% \begin{tikzfigure}%[Overall architecture of \emph{GOAM} algorithm]
%     \missingfigure[figcolor=white]{Testing figcolor}
% \end{tikzfigure}
%   where $G_q$ is the query group,
%   $n$ is the number of compare groups,
%   and $h_{k_s}$ is the histogram representation of $G_k$ in the subspace $s$.

\begin{description}
  	\item[Train Dataset description]
    There are 4446966 rows and 29
    columns.
    4446966 unique ID.
    2026745 unique groupId
    \item [Test Dataset description]
    There are 1934174 rows and 28
    columns.
    1934174 unique ID
    886238 unique groupId
\end{description}
}
%%%%%%%%%% -------------------------------------------------------------------- %%%%%%%%%%
% Second column - first block


%%%%%%%%%% -------------------------------------------------------------------- %%%%%%%%%%
\block[titleleft]{Feature and model selection}
{
\begin{description}
  	\item[Linear regression], use grid search to look for best parameters.
\end{description}
\vspace{.5cm}
\begin{tabular}{ c | c | c  }
    \toprule
    Parameters     &  Values    & CV     \\
    \midrule
    fit_intercept       &  True/False  &  3       \\

     normalize &  True/False   &  3     \\
     \bottomrule
\end{tabular}
\vspace{.2cm}
\begin{description}
    \item
    [Decision Tree], use grid search to look for best parameters.
\end{description}
\vspace{.5cm}
\begin{tabular}{ c | c | c }
    \toprule
    Parameters     &  Values    & CV     \\
    \midrule
    criterion       &  "mse", "friedman_mse", "mae"   &  3       \\

     min_samples_leaf &  1,2   &  3     \\
    \bottomrule
\end{tabular}
           
% \begin{minipage}{0.5\linewidth}
%     \centering
%     \begin{tikzfigure}
%     \missingfigure[figcolor=white]{Testing figcolor}

%     {\small{New Orleans Pelicans on FT\%}}
%     \end{tikzfigure}%
% \end{minipage}
% \hfill
% \begin{minipage}{0.5\linewidth}
%     \centering
%     \begin{tikzfigure}
%     \missingfigure[figcolor=white]{Testing figcolor}

%     {\small{New Orleans Pelicans on FTA}}
%     \end{tikzfigure}%
% \end{minipage}
% \vspace{.2cm}
% \begin{description}
% \item
% \texttt{New Orleans Pelicans} has more players with
% lower \{free throw percentage\}, \{free throws attempted\}.
% \end{description}
}
%%%%%%%%%% -------------------------------------------------------------------- %%%%%%%%%%


% Second column - second block
%%%%%%%%%% -------------------------------------------------------------------- %%%%%%%%%%
\block[titlewidthscale=1, bodywidthscale=1]
{Conclusion}
{
\begin{description}
  \item Both training and testing data shows that linear model get lower mean square error
  value.
  \item Most players choose to play squad-fpp and duo-fpp

  \item More walking distance always can bring higher win place
\end{description}
}
%%%%%%%%%% -------------------------------------------------------------------- %%%%%%%%%%


% Bottomblock
%%%%%%%%%% -------------------------------------------------------------------- %%%%%%%%%%
\colorlet{notebgcolor}{blue!20}
\colorlet{notefrcolor}{blue!20}
\note[targetoffsetx=8cm, targetoffsety=-4cm, angle=30, rotate=15,
radius=2cm, width=.26\textwidth]{
Acknowledgement
\begin{itemize}
    \item
    International Cooperation Project (Y7Z0511101)
    of IIE,
    Chinese Academy of Sciences
 \end{itemize}
}

%\note[targetoffsetx=8cm, targetoffsety=-10cm,rotate=0,angle=180,radius=8cm,width=.46\textwidth,innersep=.1cm]{
%Acknowledgement
%}

%\block[titlewidthscale=0.9, bodywidthscale=0.9]
%{Acknowledgement}{
%}
%%%%%%%%%% -------------------------------------------------------------------- %%%%%%%%%%

\end{columns}


%%%%%%%%%% -------------------------------------------------------------------- %%%%%%%%%%
%[titleleft, titleoffsetx=2em, titleoffsety=1em, bodyoffsetx=2em,%
%roundedcorners=10, linewidth=0mm, titlewidthscale=0.7,%
%bodywidthscale=0.9, titlecenter]

%\colorlet{noteframecolor}{blue!20}
\colorlet{notebgcolor}{blue!20}
\colorlet{notefrcolor}{blue!20}
\note[targetoffsetx=-13cm, targetoffsety=-12cm,rotate=0,angle=180,radius=8cm,width=.96\textwidth,innersep=.4cm]
{
\begin{minipage}{0.3\linewidth}
\centering
\includegraphics[width=24cm]{logos/tulip-wordmark.eps}
\end{minipage}
\begin{minipage}{0.7\linewidth}
{ \centering
 The $11^{th}$ International Conference on Knowledge Science,
  Engineering and Management (KSEM 2018),
  17-19/08/2018, Changchun, China
}
\end{minipage}
}
%%%%%%%%%% -------------------------------------------------------------------- %%%%%%%%%%


\end{document}

%\endinput
%%
%% End of file `tikzposter-template.tex'.
